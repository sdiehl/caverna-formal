\chapter{Sensitivity Analysis}
\label{chap:sensitivity}

The point-estimate payoff matrix proves weak dominance conditional on 64
exact values.  This chapter strengthens the result: we replace each scalar
entry with a closed interval bounding the true payoff, and prove that
$\FurnRush{}$ remains weakly dominant for \emph{every} matrix consistent
with these bounds.

\section{Interval payoff matrix}

\begin{definition}[Payoff interval]
  \label{def:payoff_interval}
  \lean{Caverna.PayoffInterval}
  \leanok
  A payoff interval $[l, h]$ with $l \le h$ bounds the true payoff
  for a particular (row, column) matchup.  The point estimate from
  \texttt{payoffMatrix} satisfies $l \le M_{i,j} \le h$.
\end{definition}

\begin{definition}[Interval payoff matrix]
  \label{def:interval_payoff}
  \lean{Caverna.intervalPayoff}
  \leanok
  \uses{def:payoff_interval}
  The interval payoff matrix assigns a $\text{PayoffInterval}$
  to each cell.  Error bounds are asymmetric and per-cell:
  \begin{itemize}
    \item Mirror matchups (diagonal): $\varepsilon = 2$.
    \item Cross-archetype matchups: $\varepsilon = 5$.
    \item Weapon rush vs.\ furnishing rush: $\varepsilon = 3$.
  \end{itemize}
\end{definition}

\begin{theorem}[Point estimates contained]
  \label{thm:point_estimates_contained}
  \lean{Caverna.point_estimates_contained}
  \leanok
  \uses{def:interval_payoff, def:payoff_matrix}
  For all $i, j$, the point estimate $M_{i,j}$ lies within the
  interval $[\text{lo}(i,j),\, \text{hi}(i,j)]$.
\end{theorem}

\section{Robust weak dominance}

\begin{theorem}[Robust weak dominance]
  \label{thm:robust_weak_dominance}
  \lean{Caverna.robust_weak_dominance}
  \leanok
  \uses{def:interval_payoff}
  For every column $j$ and every non-furnishing alternative row $i \ne 0$,
  $$
    \text{lo}(0, j) \;\ge\; \text{hi}(i, j).
  $$
  Equivalently: for any true payoff matrix within the intervals,
  $\FurnRush{}$ weakly dominates every alternative.
\end{theorem}

\begin{definition}[Robust margin]
  \label{def:robust_margin}
  \lean{Caverna.robustMarginInCol}
  \leanok
  \uses{def:interval_payoff}
  The robust margin in column $j$ is
  $\text{lo}(0,j) - \max_{i > 0}\,\text{hi}(i,j)$.
  A non-negative margin implies robust weak dominance in that column.
\end{definition}

\begin{theorem}[Tightest margin]
  \label{thm:tightest_robust_margin}
  \lean{Caverna.tightest_robust_margin}
  \leanok
  \uses{def:robust_margin}
  The tightest robust margin is $0$, occurring in column~$0$
  (mirror matchup): $\FurnRush{}$ interval $[83, 87]$ meets
  $\WeapRush{}$ interval $[77, 83]$ at the boundary $83 \ge 83$.
\end{theorem}

\begin{theorem}[All margins non-negative]
  \label{thm:all_robust_margins_nonneg}
  \lean{Caverna.all_robust_margins_nonneg}
  \leanok
  \uses{def:robust_margin}
  Every column has a non-negative robust margin.
\end{theorem}

\begin{theorem}[Non-mirror robustness]
  \label{thm:non_mirror_columns_robust}
  \lean{Caverna.non_mirror_columns_robust}
  \leanok
  \uses{def:robust_margin}
  All columns except the mirror ($j \ne 0$) have a robust margin
  of at least $20$ points, making them resilient to substantially
  larger estimation errors.
\end{theorem}

\section{Robust Nash equilibrium and welfare}

\begin{theorem}[Robust Nash equilibrium]
  \label{thm:robust_nash}
  \lean{Caverna.robust_nash_equilibrium}
  \leanok
  \uses{thm:robust_weak_dominance}
  $(\FurnRush{}, \FurnRush{})$ is a Nash equilibrium for every
  payoff matrix consistent with the interval bounds.
\end{theorem}

\begin{theorem}[Robust Nash welfare]
  \label{thm:robust_nash_welfare}
  \lean{Caverna.robust_nash_welfare_bounds}
  \leanok
  \uses{def:interval_payoff}
  The Nash welfare lies in $[166, 174]$.  The point estimate
  $170$ is contained in this interval.
\end{theorem}

\begin{theorem}[Robust social optimum]
  \label{thm:robust_social_optimum}
  \lean{Caverna.robust_social_optimum_bounds}
  \leanok
  \uses{def:interval_payoff}
  The social optimum candidate ($\FurnRush{}$ vs.\ $\AnimHusb{}$)
  has welfare in $[200, 220]$.
\end{theorem}

\begin{theorem}[Robust price of anarchy]
  \label{thm:robust_poa}
  \lean{Caverna.robust_price_of_anarchy}
  \leanok
  \uses{thm:robust_nash_welfare, thm:robust_social_optimum}
  Even in the best case for Nash welfare ($174$) and worst case for the
  social optimum ($200$), the social optimum exceeds Nash welfare.
  The price of anarchy is robust: selfish play always costs welfare.
\end{theorem}

\section{Error bound analysis}

\begin{theorem}[Estimates within bounds]
  \label{thm:estimates_within_error_bounds}
  \lean{Caverna.estimates_within_error_bounds}
  \leanok
  \uses{def:interval_payoff}
  Every point estimate differs from its interval boundary by at most
  $\varepsilon_{\max} = 5$ points.
\end{theorem}

\begin{theorem}[Column 0 fragility]
  \label{thm:fragility_column_0}
  \lean{Caverna.fragility_column_0}
  \leanok
  \uses{def:interval_payoff}
  If all error bounds were widened by $1$, robust dominance would fail
  in column~$0$ (furnishing rush lower bound $82$ vs.\ weapon rush
  upper bound $84$).  This precisely quantifies the fragility of the
  mirror-column result.
\end{theorem}
