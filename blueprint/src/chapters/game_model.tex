\chapter{Game Model}

This chapter establishes the formal model of 2-player Caverna as a finite
extensive-form game.  We define the core types (resources, animals, crops,
weapons, dwarfs), the player state, the board geometry, and the game state
machine.

\section{Core types}

\begin{definition}[Resources and animals]
  \label{def:core_types}
  \lean{Caverna.Resource, Caverna.AnimalType, Caverna.FarmAnimalType, Caverna.CropType}
  \leanok
  The game has seven resource types (wood, stone, ore, gold, grain, vegetables, food),
  four animal types (dogs, sheep, wild boar, cattle, where dogs are non-farm animals),
  and two crop types (grain, vegetables).
\end{definition}

\begin{definition}[Weapons and dwarfs]
  \label{def:weapons_dwarfs}
  \lean{Caverna.Weapon, Caverna.Dwarf}
  \leanok
  A weapon has integer strength in $[1,14]$.  A dwarf optionally carries a weapon.
  Each player begins with $2$ dwarfs (Definition~\ref{def:initial_state}).
\end{definition}

\begin{definition}[Player state]
  \label{def:player_state}
  \lean{Caverna.PlayerState}
  \leanok
  \uses{def:core_types, def:weapons_dwarfs}
  A player state tracks: the list of dwarfs, resource supplies
  (wood, stone, ore, gold, grain, vegetables, food, rubies),
  animal counts, board tile placements (mountain spaces, forest spaces,
  meadows, fields, pastures, mines), installed furnishing tiles,
  and accumulated begging markers.
\end{definition}

\begin{definition}[Initial state]
  \label{def:initial_state}
  \lean{Caverna.initialDwarfCount, Caverna.startingFoodP1, Caverna.startingFoodP2, Caverna.initialPlayerState}
  \leanok
  \uses{def:player_state}
  Each player starts with $2$ unarmed dwarfs.
  Player~1 receives $1$ food; Player~2 receives $2$ food.
  All other resources and animals start at zero.
  The initial dwarf count is $\text{initialDwarfCount} = 2$.
\end{definition}

\section{Game structure}

\begin{definition}[Action spaces]
  \label{def:action_spaces}
  \lean{Caverna.ActionSpaceId, Caverna.actionSpaceRound}
  \leanok
  The game has $24$ action spaces.  Thirteen are available from round~1
  (preprinted); the rest are revealed one per round.
  Each space can be occupied by at most one dwarf per round.
\end{definition}

\begin{definition}[Harvest schedule]
  \label{def:harvest}
  \lean{Caverna.twoPlayerHarvestSchedule, Caverna.HarvestEvent}
  \leanok
  In the 2-player game, harvests occur on a fixed schedule across 12 rounds.
  Round~3 always triggers a normal harvest (Theorem~\ref{thm:round3_harvest}).
\end{definition}

\begin{figure}[ht]
\centering
\begin{tikzpicture}[
  rnd/.style={circle, minimum size=0.6cm, inner sep=0pt, font=\tiny\bfseries,
              text=textDark, draw=textDark, thin},
  normal/.style={rnd, fill=pastelCream},
  harvest/.style={rnd, fill=pastelMint, draw=accentGreen, thick},
  milestone/.style={font=\tiny, text=accentBlue, align=center},
]
% Round nodes
\foreach \r in {1,...,12} {
  \node[normal] (r\r) at (\r * 0.9, 0) {\r};
}
% Mark harvest rounds (depends on 2p schedule; rounds 3,5,7,9,10,11,12 are common)
% From the Lean source: harvests at rounds 3, 5, 7, 9, 10, 11, 12
\foreach \r in {3,5,7,9,10,11,12} {
  \node[harvest] (r\r) at (\r * 0.9, 0) {\r};
}
% Arrows
\foreach \r [evaluate=\r as \t using int(\r+1)] in {1,...,11} {
  \draw[-{Stealth[length=2.5pt]}, thin, textDark] (r\r) -- (r\t);
}
% Milestones
\draw[accentBlue, thin, dashed] (4*0.9, -0.45) -- (4*0.9, -1.0);
\node[milestone] at (4*0.9, -1.3) {Wish for\\Children};
\draw[accentBlue, thin, dashed] (8*0.9, -0.45) -- (8*0.9, -1.0);
\node[milestone] at (8*0.9, -1.3) {Family\\Life};
% Action space count annotation
\draw[accentAmber, thick, decorate, decoration={brace, amplitude=4pt}]
  (1*0.9 - 0.35, 0.55) -- (12*0.9 + 0.35, 0.55)
  node[midway, above=5pt, font=\tiny, text=accentAmber] {13 preprinted + 1 new per round = 24 total action spaces};
% Legend
\node[normal, label={[font=\tiny, text=textDark]right:normal round}] at (2.5, -2.2) {};
\node[harvest, label={[font=\tiny, text=textDark]right:harvest round (feeding)}] at (6.5, -2.2) {};
\end{tikzpicture}
\caption{Game timeline for 2-player Caverna.  Green nodes mark harvest rounds
where dwarfs must be fed.  Key milestones: ``Wish for Children'' (round~4) and
``Family Life'' (round~8) enable family growth.  One new action space is revealed
each round, growing from $13$ to $24$ available spaces.}
\label{fig:timeline}
\end{figure}

\begin{definition}[Board geometry]
  \label{def:board}
  \lean{Caverna.totalBoardSpaces}
  \leanok
  The personal player board has $24$ spaces total
  (Theorem~\ref{thm:board_24}).
\end{definition}

\begin{figure}[ht]
\centering
\begin{tikzpicture}[
  state/.style={rounded corners=5pt, minimum width=2.0cm, minimum height=0.7cm,
                font=\scriptsize, text=textDark, draw=textDark, thin, align=center},
  trans/.style={-{Stealth[length=5pt]}, thick, color=textDark},
  act/.style={font=\tiny, text=accentTeal, fill=white, inner sep=1.5pt},
]
% States
\node[state, fill=pastelCream] (s0) at (0, 0) {Start of round\\2 dwarfs, 1 food};
\node[state, fill=pastelSky]   (s1) at (4.5, 1.2) {Place dwarf 1\\on Logging};
\node[state, fill=pastelSky]   (s2) at (4.5, -1.2) {Place dwarf 1\\on Excavation};
\node[state, fill=pastelLemon] (s3) at (9.0, 1.2) {+3 wood\\Place dwarf 2};
\node[state, fill=pastelLemon] (s4) at (9.0, -1.2) {+1 stone, tunnel\\Place dwarf 2};
\node[state, fill=pastelMint]  (s5) at (13.0, 0) {All placed\\$\to$ Harvest?};
% Transitions
\draw[trans] (s0) -- (s1) node[act, midway, above, sloped] {Logging};
\draw[trans] (s0) -- (s2) node[act, midway, below, sloped] {Excavation};
\draw[trans] (s1) -- (s3) node[act, midway, above] {collect};
\draw[trans] (s2) -- (s4) node[act, midway, above] {collect};
\draw[trans] (s3) -- (s5) node[act, midway, above, sloped] {place dwarf 2};
\draw[trans] (s4) -- (s5) node[act, midway, below, sloped] {place dwarf 2};
% Dots indicating more branches
\node[font=\large, text=pastelSlate] at (4.5, 0) {$\vdots$};
\node[font=\tiny, text=pastelSlate, anchor=west] at (5.0, 0) {11 other actions};
\end{tikzpicture}
\caption{Fragment of the game state machine (labelled transition system) for one
round.  Each state records the player's resources and remaining dwarfs to place.
Transitions correspond to action space selections.  With $13$ initial action
spaces and $2$ dwarfs, even a single round produces $13 \times 12 = 156$
placement sequences.}
\label{fig:lts}
\end{figure}

\begin{theorem}[Board has 24 spaces]
  \label{thm:board_24}
  \lean{Caverna.board_has_24_spaces}
  \leanok
  \uses{def:board}
  $\text{totalBoardSpaces} = 24$.
\end{theorem}

\begin{proof}
  \leanok
  By computation.
\end{proof}
