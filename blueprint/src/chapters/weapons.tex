\chapter{Weapon System}

Weapons enable expeditions, which provide loot (resources, animals,
furnishing actions).  This chapter formalizes weapon forging, expedition
strength growth, and the loot table, which are needed to evaluate the
weapon rush archetype.

\section{Forging and growth}

\begin{definition}[Weapon forging]
  \label{def:forge}
  \lean{Caverna.forgeWeapon, Caverna.upgradeWeapon}
  \leanok
  A weapon is forged at strength $s \in [1,8]$ by spending $s$ ore.
  After each expedition, the weapon's strength increases by $1$,
  up to a cap of $14$.
\end{definition}

\begin{theorem}[Maximum forge strength]
  \label{thm:max_forge}
  \lean{Caverna.max_forge_strength, Caverna.cannot_forge_above_8}
  \leanok
  \uses{def:forge}
  A weapon can be forged at strength $8$ (costing $8$ ore).
  No weapon can be forged above strength $8$.
\end{theorem}

\begin{proof}\leanok By the forging rules: valid range is $[1,8]$.\end{proof}

\begin{theorem}[Expedition increases strength]
  \label{thm:exp_increase}
  \lean{Caverna.expedition_increases_strength, Caverna.weapon_cap}
  \leanok
  \uses{def:forge}
  If $s < 14$, an expedition raises strength to $s+1$.
  At $s = 14$, strength stays at $14$.
\end{theorem}

\begin{proof}\leanok Direct from the upgrade function definition.\end{proof}

\begin{figure}[ht]
\centering
\begin{tikzpicture}[
  str/.style={circle, minimum size=0.55cm, inner sep=0pt, font=\tiny\bfseries,
              text=textDark, draw=textDark, thin},
  forge/.style={str, fill=pastelSky},
  exp/.style={str, fill=pastelPeach},
  maxcap/.style={str, fill=pastelCoral, draw=accentAmber, thick},
  arr/.style={-{Stealth[length=3pt]}, thin, color=textDark},
]
% Forge range: 1-8
\foreach \s in {1,...,8} {
  \node[forge] (s\s) at (\s * 0.85, 0) {\s};
}
% Expedition range: 9-13
\foreach \s in {9,...,13} {
  \node[exp] (s\s) at (\s * 0.85, 0) {\s};
}
% Cap: 14
\node[maxcap] (s14) at (14 * 0.85, 0) {14};
% Arrows between consecutive
\foreach \s [evaluate=\s as \t using int(\s+1)] in {1,...,13} {
  \draw[arr] (s\s) -- (s\t);
}
% Forge bracket
\draw[accentBlue, thick, decorate, decoration={brace, amplitude=4pt, mirror}]
  (0.85*1 - 0.35, -0.5) -- (0.85*8 + 0.35, -0.5)
  node[midway, below=5pt, font=\tiny, text=accentBlue] {Forge range (1--8 ore)};
% Expedition bracket
\draw[accentAmber, thick, decorate, decoration={brace, amplitude=4pt, mirror}]
  (0.85*9 - 0.35, -0.5) -- (0.85*14 + 0.35, -0.5)
  node[midway, below=5pt, font=\tiny, text=accentAmber] {Expedition only};
% Cattle threshold
\draw[accentTeal, thick, dashed] (0.85*9 - 0.15, 0.55) -- (0.85*9 - 0.15, 0.9);
\node[font=\tiny, text=accentTeal, anchor=south] at (0.85*9, 0.95) {Cattle loot $\geq 9$};
% Legend
\node[forge, label={[font=\tiny, text=textDark]right:forgeable}] at (1.0, -1.4) {};
\node[exp, label={[font=\tiny, text=textDark]right:expedition growth}] at (4.0, -1.4) {};
\node[maxcap, label={[font=\tiny, text=textDark]right:cap}] at (7.5, -1.4) {};
\end{tikzpicture}
\caption{Weapon strength growth path.  Weapons are forged at strength $1$--$8$
(costing that many ore), then grow by $+1$ per expedition up to the cap of $14$.
Cattle loot requires reaching strength $\geq 9$, which needs at least one expedition
after forging.}
\label{fig:weapon_path}
\end{figure}

\begin{theorem}[Expeditions to max]
  \label{thm:exp_to_max}
  \lean{Caverna.expeditions_from_max_forge, Caverna.expeditions_from_min_forge}
  \leanok
  \uses{thm:exp_increase, thm:max_forge}
  Starting from strength $8$, it takes $6$ expeditions to reach $14$.
  Starting from strength $1$, it takes $13$.
\end{theorem}

\begin{proof}\leanok $14 - 8 = 6$ and $14 - 1 = 13$.\end{proof}

\section{Loot table}

\begin{definition}[Available loot]
  \label{def:loot}
  \lean{Caverna.availableLootCount, Caverna.ExpeditionLevel}
  \leanok
  The number of loot options available depends on weapon strength.
  Higher strength unlocks more valuable loot items.
\end{definition}

\begin{theorem}[Loot at key strengths]
  \label{thm:loot_values}
  \lean{Caverna.loot_at_strength_1, Caverna.loot_at_strength_8, Caverna.loot_at_strength_14}
  \leanok
  \uses{def:loot}
  At strength $1$: $3$ loot options.
  At strength $8$: $13$ options.
  At strength $14$: $18$ options.
\end{theorem}

\begin{proof}\leanok By enumeration of the loot table.\end{proof}

\begin{theorem}[Loot monotonicity]
  \label{thm:loot_monotone}
  \lean{Caverna.loot_monotone_small}
  \leanok
  \uses{def:loot}
  $\text{availableLootCount}(1) \le \text{availableLootCount}(8)$.
\end{theorem}

\begin{proof}\leanok $3 \le 13$.\end{proof}

\begin{theorem}[Premium loot count]
  \label{thm:premium_loot}
  \lean{Caverna.premium_loot_count}
  \leanok
  \uses{def:loot}
  $\text{availableLootCount}(14) - \text{availableLootCount}(8) = 5$.
  The last $5$ loot items require strength above $8$.
\end{theorem}

\begin{proof}\leanok $18 - 13 = 5$.\end{proof}

\begin{theorem}[Cattle requires expedition]
  \label{thm:cattle_expedition}
  \lean{Caverna.cattle_requires_expedition}
  \leanok
  \uses{def:loot}
  Cattle's minimum required strength exceeds the maximum initial
  weapon strength, meaning cattle can only be obtained through expeditions.
\end{theorem}

\begin{proof}\leanok Cattle requires strength $\ge 9$, but weapons forge at $\le 8$.\end{proof}
